\documentclass[12pt]{article}

\usepackage[T2A]{fontenc}
\usepackage[utf8]{inputenc}
\usepackage{cancel}

%\usepackage[leqno]{amsmath}
\pagestyle{empty}

\usepackage{xcolor}
\usepackage{hyperref}

% Цвета для гиперссылок
\definecolor{linkcolor}{HTML}{799B03} % цвет ссылок
\definecolor{urlcolor}{HTML}{799B03} % цвет гиперссылок

\hypersetup{pdfstartview=FitH,  linkcolor=linkcolor,urlcolor=urlcolor, colorlinks=true}


\usepackage{tikz}
\usepackage{pgfplots}

\usepackage{tikz}
\usepackage{chngcntr}

\usepackage[english,russian]{babel}	% локализация и переносы

%%% Дополнительная работа с математикой
\usepackage{amsmath,amsfonts,amssymb,amsthm,mathtools}


\usepackage{icomma} % "Умная" запятая: $0,2$ --- число, $0, 2$ --- перечисление

\vspace{5mm}

\usepackage{euscript}	 % Шрифт Евклид
\usepackage{mathrsfs} % Красивый матшрифт

\parskip=1pt %интервал между абзацами
\oddsidemargin=-1.3cm %отступ от правого края
\topmargin=-2.5cm %отступ от верхнего края
\textwidth=18.5cm %Ширина текста
\textheight=24cm

\newcommand*{\hm}[1]{#1\nobreak\discretionary{}
	{\hbox{$\mathsurround=0pt #1$}}{}}

\renewcommand{\baselinestretch}{1.25}%Расстояние между строками


\begin{document}
	
	{\tt Пример 4.} Найти
	\[
		\lim\limits_{x\to0}\dfrac{\sqrt{1+x}-1}{\sqrt[3]{x+1} -1}
	\]
	{\tt Решение.} Полагая, что $1+x=y^6$ имеем:
	\[
		\lim\limits_{x\to0}\dfrac{\sqrt{1+x}-1}{\sqrt[3]{x+1}-1} = \lim\limits_{y\to1} \dfrac{y^3-1}{y^2-1} = \lim\limits_{y\to1} \dfrac{y^2+y+1}{y+1} = \dfrac{3}{2}
	\]
	
	
	\medskip
	{\bf 199.} $\lim\limits_{x\to1}\dfrac{\sqrt{x}-1}{x-1}$. Полагая, что $y=\sqrt{x}$ имеем: $\lim\limits_{y\to1} \dfrac{y-1}{y^2-1} = \lim\limits_{y\to1} \dfrac{1}{y+1} = \dfrac{1}{2}$
	
	\medskip
	{\bf 200.} $\lim\limits_{x\to64} \dfrac{\sqrt{x}-8}{\sqrt[3]{x}-4}$. Полагая, что $y=\sqrt[6]{x}$ имеем: $\lim\limits_{y\to2}\dfrac{y^3-8}{y^2-4} = \lim\limits_{y\to2} \dfrac{y^2+2y+2}{y+2} = \dfrac{4+4+2}{2+2} = \dfrac{5}{2}$
	
	\medskip
	{\bf 201.} $\lim\limits_{x\to1} \dfrac{\sqrt[3]{x}-1}{\sqrt[4]{x}-1}$. Полагая, что $y=\sqrt[12]{x}$ имеем: $\lim\limits_{y\to1}\dfrac{y^4-1}{y^3-1} = \lim\limits_{y\to1} \dfrac{(y-1)(y+1)(y^2+1)}{(y-1)(y^2+y+1)} =$
	
	$
	=\lim\limits_{y\to1}\dfrac{(y+1)(y^2+1)}{y^2+y+1} = \dfrac{2\cdot2}{1+1+1} = \dfrac{4}{3}
	$
	
	\medskip
	{\bf 202.} $\lim\limits_{x\to1}\dfrac{\sqrt[3]{x^2}-2\sqrt[3]{x}+1}{(x-1)^2}$. Полагая, что $y=\sqrt[3]{x}$ имеем: $\lim\limits_{y\to1} \dfrac{y^2-2y+1}{(y^3-1)^2} = \lim\limits_{y\to1}\dfrac{(y-1)^2}{(y-1)^2(y^2+y+1)^2} = $
	
	$
	= \lim\limits_{y\to1} \dfrac{1}{(y^2+y+1)^2} = \dfrac{1}{(1+1+1)^2} = \dfrac{1}{3^2} = \dfrac{1}{9}
	$
	
	
	\medskip
	{\bf 203.} $\lim\limits_{x\to7} \dfrac{2-\sqrt{x-3}}{x^2-49} = \lim\limits_{x\to7} \dfrac{4-x+3}{(x-7)(x+7)(2+\sqrt{x-3})} = -\lim\limits_{x\to7} \dfrac{1}{(x+7)(2+\sqrt{x-3})} = -\dfrac{1}{56}$
	
	\medskip
	{\bf 204.} $\lim\limits_{x\to8}\dfrac{x-8}{\sqrt[3]{x}-2} = \lim\limits_{x\to8} \dfrac{(x-8)(\sqrt[3]{x^2}+2\sqrt[3]{x}+4)}{x-8} = \lim\limits_{x\to8} (\sqrt[3]{x^2}+2\sqrt[3]{x}+4) = 2^2+2\cdot2+4 = 12$
	
	\medskip
	{\bf 205.} $\lim\limits_{x\to1}\dfrac{\sqrt{x}-1}{\sqrt[3]{x}-1} = \lim\limits_{x\to1} \dfrac{(x-1)(\sqrt[3]{x^2}+\sqrt[3]{x}+1)}{(x-1)(\sqrt{x}+1)} = \lim\limits_{x\to1} \dfrac{\sqrt[3]{x^2}+\sqrt[3]{x}+1}{\sqrt{x}+1} = \dfrac{3}{2}$
	
	\medskip
	{\bf 206.} $\lim\limits_{x\to4} \dfrac{3-\sqrt{5+x}}{1-\sqrt{5-x}} = \lim\limits_{x\to4} \dfrac{(9-5-x)(1+\sqrt{5-x})}{(1-5+x)(3+\sqrt{5+x})} = -\lim\limits_{x\to4} \dfrac{(x-4)(1+\sqrt{5-x})}{(x-4)(3+\sqrt{5+x})} = -\dfrac{1+1}{3+3} = -\dfrac{1}{3}$
	
	\medskip
	{\bf 207.} $\lim\limits_{x\to0} \dfrac{\sqrt{1+x}-\sqrt{1-x}}{x} = \lim\limits_{x\to0} \dfrac{1+x-1+x}{x(\sqrt{1+x}+\sqrt{1-x})} = \lim\limits_{x\to0} \dfrac{2}{\sqrt{1+x}+\sqrt{1-x}} = 1$
	
	\medskip
	{\bf 208.} $\lim\limits_{h\to0} \dfrac{\sqrt{x+h}-\sqrt{x}}{h} = \lim\limits_{h\to0} \dfrac{x+h-x}{h(\sqrt{x+h}+\sqrt{x})} = \lim\limits_{h\to0} \dfrac{1}{\sqrt{x+h}+\sqrt{x}} = \dfrac{1}{2\sqrt{x}}$
	
	\medskip
	{\bf 209.} $\lim\limits_{h\to0} \dfrac{\sqrt[3]{x+h}-\sqrt[3]{x}}{h} = \lim\limits_{h\to0} \dfrac{x+h-x}{h(\sqrt[3]{(x+h)^2}+\sqrt[3]{x(x+h)}+\sqrt[3]{x^2})} = \lim\limits_{h\to0} \dfrac{1}{\sqrt[3]{(x+h)^2}+\sqrt[3]{x(x+h)}+\sqrt[3]{x^2}} =$
	
	$=\dfrac{1}{\sqrt[3]{(x)^2}+\sqrt[3]{x(x)}+\sqrt[3]{x^2}} = \dfrac{1}{3\sqrt[3]{x^2}}$
	
	\medskip
	{\bf 210.} $\lim\limits_{x\to3} \dfrac{\sqrt{x^2-2x+6}-\sqrt{x^2+2x-6}}{x^2-4x+3} = \lim\limits_{x\to3} \dfrac{x^2-2x+6-x^2-2x+6}{(x^2-4x+3)(\sqrt{x^2-2x+6}+\sqrt{x^2+2x-6})} = $
	
	$
	=-\lim\limits_{x\to3}\dfrac{4(x-3)}{(x-3)(x-1)(\sqrt{x^2-2x+6}+\sqrt{x^2+2x-6})} = -\lim\limits_{x\to3} \dfrac{4}{(x-1)(\sqrt{x^2-2x+6}+\sqrt{x^2+2x-6})} = 
	$
	
	$
	= \dfrac{-4}{2(\sqrt{9-6+6}+\sqrt{9+6-6})} = \dfrac{-2}{3+3} = -\dfrac{1}{3}
	$
	
	\medskip
	{\bf 211.} $\lim\limits_{x\to+\infty} (\sqrt{x+a}-\sqrt{x}) = \lim\limits_{x\to+\infty} \dfrac{x+a-x}{\sqrt{x+a}+\sqrt{x}} = \lim\limits_{x\to+\infty} \dfrac{a}{\sqrt{x+a}+\sqrt{x}} = 0$
	
	\medskip
	{\bf 212.} $\lim\limits_{x\to+\infty} (\sqrt{x(x+a)}-x) = \lim\limits_{x\to+\infty} \dfrac{x^2+ax-x^2}{\sqrt{x(x+a)}+x} = \lim\limits_{x\to+\infty} \dfrac{a}{\sqrt{1+\frac{a}{x}}+1} = \dfrac{a}{2}$
	
	\medskip
	{\bf 213.} $\lim\limits_{x\to+\infty}(\sqrt{x^2-5x+6}-x) = \lim\limits_{x\to+\infty} \dfrac{x^2-5x+6-x^2}{\sqrt{x^2-5x+6}+x} = \lim\limits_{x\to+\infty} \dfrac{\frac{6}{x}-5}{\sqrt{1-\frac{5}{x}+\frac{6}{x^2}}+1} = \dfrac{-5}{2}$
	
	\medskip
	{\bf 214.} $\lim\limits_{x\to+\infty} x(\sqrt{x^2+1}-x) = \lim\limits_{x\to+\infty} \dfrac{x(x^2+1-x^2)}{\sqrt{x^2+1}+x} = \lim\limits_{x\to+\infty} \dfrac{1}{\sqrt{1+\frac{1}{x^2}}+1} = \dfrac{1}{2}$

	\medskip
	{\bf 215.} $\lim\limits_{x\to\infty}(x+\sqrt[3]{1-x^3}) = \lim\limits_{x\to\infty} \dfrac{x^3+1-x^3}{x^2+x\sqrt[3]{1-x^3}+\sqrt[3]{(1-x^3)^2}} = \lim\limits_{x\to\infty} \dfrac{1}{x^2+x\sqrt[3]{1-x^3}+\sqrt[3]{(1-x^3)^2}} = $
	
	$
	=\lim\limits_{x\to\infty} \dfrac{\frac{1}{x^2}}{1+\sqrt[3]{\frac{1}{x^3}-1}+\sqrt[3]{(\frac{1}{x^3}-1)^2}} = 0
	$
	
	\medskip
	
	При вычислении пределов во многих случаях используется формула. Первый замечательный предел.
	\[
		\lim\limits_{x\to 0} \dfrac{\sin x}{x} = 1
	\]
	и предполагается известным, что $\lim\limits_{x\to a}\sin x = \sin a$ и $\lim\limits_{x\to a}\cos x = \cos a$.
	
	
	\medskip
	{\bf 216.} а) $\lim\limits_{x\to2} \dfrac{\sin x}{x} = \dfrac{\sin 2}{2}$
	
	\quad б) $\lim\limits_{x\to\infty} \dfrac{\sin x}{x}$ , т.к\quad $-1\le \sin x \le 1$, то 
	\[
		\lim\limits_{x\to\infty}\dfrac{-1}{x}=0 \le \lim\limits_{x\to\infty} \dfrac{\sin x}{x} \le \lim\limits_{x\to\infty} \dfrac{1}{x}=0
	\]
	поэтому $\lim\limits_{x\to\infty} \dfrac{\sin x}{x} = 0$
	
	\medskip
	{\bf 217.} $\lim\limits_{x\to0} \dfrac{\sin 3x}{x} = 3\lim\limits_{3x\to0} \dfrac{\sin (3x)}{3x} = 3\cdot1 = 3$
	
	\medskip
	{\bf 218.} $\lim\limits_{x\to0} \dfrac{\sin 5x}{\sin 2x} = \dfrac{5}{2}\lim\limits_{x\to0} \dfrac{\frac{\sin 5x}{5x}}{\frac{\sin 2x}{2x}} = \dfrac{5}{2}\cdot \dfrac{1}{1} = \dfrac{5}{2}$
	
	\medskip
	{\bf 219.} $\lim\limits_{x\to1}$. Полагая, что $y=x-1$ имеем: $\lim\limits_{y\to0} \dfrac{\sin(\pi(y+1))}{\sin(3\pi(y+1))} = \lim\limits_{y\to0} \dfrac{\sin \pi y}{\sin 3\pi y} = \dfrac{1}{3}\dfrac{\frac{\sin \pi y}{\pi y}}{\frac{\sin 3\pi y}{3\pi y}} = \dfrac{1}{3}$
	
	\medskip
	{\bf 220.} $\lim\limits_{x\to\infty} \left(n\sin\dfrac{\pi}{n}\right) = \pi\lim\limits_{x\to\infty} \dfrac{\sin \frac{\pi}{n}}{\frac{\pi}{n}} = \pi \lim\limits_{y\to 0}\dfrac{\sin \pi y}{\pi y} = \pi$
	
	\medskip
	{\bf 221.} $\lim\limits_{x\to0} \dfrac{1-\cos x}{x^2} = \lim\limits_{x\to0} \dfrac{1-\cos^2 x}{x^2(1+\cos x)} = \lim\limits_{x\to0} \dfrac{\sin^2 x}{x^2(1+\cos x)} = \dfrac{1}{2}$
	
	\medskip
	{\bf 222.} $\lim\limits_{x\to a} \dfrac{\sin x-\sin a}{x-a} = \dfrac{1}{2}\lim\limits_{x\to a} \dfrac{\sin(\frac{x-a}{2})\cos(\frac{x+a}{2})}{\frac{x-a}{2}} = \dfrac{1}{2}\lim\limits_{x\to a} \cos \frac{x+a}{2} = \dfrac{1}{2}\cos a$
	
	\medskip
	{\bf 223.} $\lim\limits_{x\to a} \dfrac{\cos x-\cos a}{x-a} = -\lim\limits_{x\to a} \dfrac{\sin\frac{x-a}{2}\sin\frac{x+a}{2}}{x-a} = -\dfrac{1}{2}\lim\limits_{x\to a} \sin\frac{x+a}{2} = -\dfrac{1}{2}\sin a$
	
	\medskip
	{\bf 224.} $\lim\limits_{x\to-2} \dfrac{\tan \pi x}{x+2} = \lim\limits_{x\to -2} \dfrac{\tan \pi(x+2)}{x+2} = \lim\limits_{y\to 0} \dfrac{\tan \pi y}{y} = \pi$
	
	\medskip
	{\bf 225.} $\lim\limits_{h\to 0} = \lim\limits_{h\to 0} \dfrac{\sin(x+h)-\sin x}{h} = \lim\limits_{h\to 0}\dfrac{\sin x\cos h+ \sin h \cos x - \sin x}{h} = $
	
	$
	= \lim\limits_{h\to 0} \dfrac{\sin x(\cos h-1)}{h} + \lim\limits_{h\to 0} \dfrac{\sin h \cos x}{h} = - \lim\limits_{h\to 0} \dfrac{h\sin x(1-\cos^2 h)}{h\cdot h} + \cos x = - \lim\limits_{h\to 0} \dfrac{h\sin x\sin^2 h}{h^2}+
	$
	
	\vspace{2mm}
	$
	+\cos x = -\lim\limits_{h\to 0} h\sin x  + \cos x = \cos x
	$
	
	\medskip
	{\bf 226.} $\lim\limits_{x\to \frac{\pi}{4}} \dfrac{\sin x-\cos x}{1-\tan x} = - \lim\limits_{x\to \frac{\pi}{4}} \dfrac{\cos x(\sin x-\cos x)}{\sin x-\cos x} = - \lim\limits_{x\to \frac{\pi}{4}} \cos x = -\dfrac{\sqrt{2}}{2}$
	
	\medskip
	{\bf 227.} а) $\lim\limits_{x\to 0} x\sin\dfrac{1}{x} = 0$
	
	\quad б) $\lim\limits_{x\to\infty} x\sin \dfrac{1}{x}$. Полагая, что $y=\frac{1}{x}$ имеем: $\lim\limits_{y\to0} \dfrac{\sin y}{y}  = 1$
	
	\medskip
	{\bf 228.} $-\lim\limits_{x\to 1} (x-1)\tan\dfrac{\pi x}{2}$.  Полагая, что $y=x-1$ имеем: $-\lim\limits_{y\to 0}y\tan \dfrac{\pi (y+1)}{2} = \lim\limits_{y\to0} y\ctg\dfrac{\pi y}{2} =$
	
	$
	=\lim\limits_{y\to0} \dfrac{\cos\frac{\pi y}{2}}{\frac{\sin\frac{\pi y}{2}}{y}} = \lim\limits_{y\to0} \dfrac{2}{\pi} \dfrac{\cos \frac{\pi y}{2}}{\frac{\sin \frac{\pi y}{2}}{\frac{\pi y}{2}}} =  \dfrac{2}{\pi}
	$
	
	\medskip
	{\bf 229.} $\lim\limits_{x\to0} \ctg 2x \ctg \left(\dfrac{\pi}{2}-x\right) = \lim\limits_{x\to0} \dfrac{\cos 2x}{\sin 2x} \tg x = \lim\limits_{x\to0} \dfrac{\cos 2x \sin x}{2\sin x\cos^2 x } = \dfrac{1}{2}\lim\limits_{x\to0} \dfrac{\cos 2x}{\cos^2 x} =\dfrac{1}{2}\lim\limits_{x\to0} \dfrac{\cos 0}{\cos 0}= \dfrac{1}{2}$
	
	\medskip
	{\bf 230.} $\lim\limits_{x\to\pi}\dfrac{1-\sin\frac{x}{2}}{\pi-x} = \lim\limits_{x\to\pi} \dfrac{\sin \frac{x}{2}-1}{x-\pi}$. Полагая, что $y=x-\pi$ имеем: $\lim\limits_{y\to0} \dfrac{\sin \frac{y+\pi}{2}-1}{y} = \lim\limits_{y\to0} \dfrac{\cos \frac{y}{2}-1}{y} =$
	
	$
	= -\lim\limits_{y\to0}\dfrac{1-\cos^2\frac{y}{2}}{y(\cos\frac{y}{2}+1)} = - \lim\limits_{y\to0} \dfrac{\sin^2\frac{y}{2}}{y(\cos\frac{y}{2}+1)} = -\lim\limits_{y\to0} \dfrac{\sin \frac{y}{2}}{\cos \frac{y}{2}+1} = 0
	$
	
	\medskip
	{\bf 231.} $\lim\limits_{x\to\frac{\pi}{3}}\dfrac{1-2\cos x}{\pi -3x} = \lim\limits_{x\to\frac{\pi}{3}}\dfrac{2\cos x -1}{3(x-\frac{\pi}{3})}$. Полагая, что $y=x-\frac{\pi}{3}$ имеем: $\lim\limits_{y\to0} \dfrac{2\cos (y+\frac{\pi}{3})-1}{3y} = $
	
	$
	 = \lim\limits_{y\to0} \dfrac{2(\cos y \cos \frac{\pi}{3} - \sin y \sin \frac{\pi}{3})-1}{3y} = \lim\limits_{y\to0} \dfrac{2(\frac{1}{2}\cos y - \frac{\sqrt{3}}{2}\sin y)-1}{3y} = \lim\limits_{y\to0} \dfrac{\cos y - \sqrt{3}\sin y-1}{3y} = 
	$
	
	$
	 = \lim\limits_{y\to0} \dfrac{\cos y - 1}{3y} - \dfrac{\sqrt{3}}{3}\lim\limits_{y\to0} \dfrac{\sin y}{y} = -\lim\limits_{y\to0} \dfrac{1-\cos^2 y}{3y(\cos y+1)} - \dfrac{\sqrt{3}}{3} = -\lim\limits_{y\to0}\dfrac{\sin^2 y}{3y(\cos y+1)}-\dfrac{\sqrt{3}}{3} = 
	$
	
	$
	= -\lim\limits_{y\to0} \dfrac{\sin y}{3(\cos y + 1)} - \dfrac{\sqrt{3}}{3} = -\lim\limits_{y\to0} \dfrac{\sin 0}{3\cos 0 + 1} - \dfrac{\sqrt{3}}{3} = -\dfrac{\sqrt{3}}{3}
	$
	
	\medskip
	{\bf 232.} $\lim\limits_{x\to0}\dfrac{\cos mx- \cos nx}{x^2} = -\lim\limits_{x\to0} \dfrac{\frac{1}{2}\sin \frac{mx+nx}{2}\sin \frac{mx-nx}{2}}{x^2} = -\lim\limits_{x\to0} \dfrac{\sin (x\frac{m+n}{2})\sin (x\frac{m-n}{2})}{2x^2} = $
	
	$
	= \dfrac{n^2-m^2}{2} \lim\limits_{x\to0} \dfrac{\sin (x\frac{m+n}{2})\sin(x\frac{m-n}{2})}{x^2\frac{m-n}{2}\frac{m+n}{2}} = \dfrac{n^2-m^2}{2}
	$
	
	\medskip
	{\bf 233.} $\lim\limits_{x\to0}\dfrac{\tg x-\sin x}{x^3} = \lim\limits_{x\to0} \dfrac{\sin x(1-\cos x)}{x^3\cos x} = \lim\limits_{x\to0} \dfrac{1-\cos^2 x}{x^2\cos x} = \lim\limits_{x\to0} \dfrac{\sin^2 x}{x^2\cos x} = \lim\limits_{x\to0} \dfrac{1}{\cos x} = 1$
	
	\medskip
	{\bf 234.} $\lim\limits_{x\to0}\dfrac{\arcsin x}{x}$. Полагая, что $x=\sin y$ имеем: $\lim\limits_{\sin y\to0} \dfrac{\arcsin\sin y}{\sin y} = \lim\limits_{\sin y\to0}\dfrac{y}{\sin y}=  1$
	
	\medskip
	{\bf 235.} $\lim\limits_{x\to0}\dfrac{\arctan 2x}{\sin 3x} = \lim\limits_{x\to0} \dfrac{\arctan 2x}{3x \frac{\sin 3x}{3x}} = \lim\limits_{x\to0} \dfrac{\arctan2x}{3x}$. Полагая, что $2x=\tan y$ имеем: 
	
	$
	\dfrac{2}{3}\lim\limits_{\tan y\to0} \dfrac{\arctan\tan y}{\tan y} = \dfrac{2}{3}\lim\limits_{\tan y\to0}\dfrac{y}{\tan y} = \dfrac{2}{3}\lim\limits_{\tan y\to0} \dfrac{\cos y}{(\sin y)/y} = \dfrac{2}{3}
	$
	
	\medskip
	{\bf 236.} $\lim\limits_{x\to1}\dfrac{1-x^2}{\sin \pi x}$. Полагая, что $y=x-1$ имеем: $-\lim\limits_{y\to0}\dfrac{y(y+2)}{\sin \pi(y+1)} = \lim\limits_{y\to0}\dfrac{y(y+2)}{\sin\pi y} = $

	$
	= \dfrac{1}{\pi}\lim\limits_{y\to0} \dfrac{\pi y(y+2)}{\sin \pi y}	= \dfrac{1}{\pi}\lim\limits_{y\to0}(y+2) = \dfrac{2}{\pi}
	$	
	
	\medskip
	{\bf 237.} $\lim\limits_{x\to0}\dfrac{x-\sin 2x}{x+\sin 3x} = \lim\limits_{x\to0} \dfrac{1-2\frac{\sin 2x}{2x}}{1+3\frac{\sin 3x}{3x}} = \dfrac{1-2\cdot1}{1+3\cdot1} = \dfrac{1-2}{1+3} = -\dfrac{1}{4}$
	
	\medskip
	{\bf 238.} $\lim\limits_{x\to1}\dfrac{\cos \frac{\pi x}{2}}{1-\sqrt{x}} = \lim\limits_{x\to1} \dfrac{\cos \frac{\pi x}{2}(1+\sqrt{x})}{1-x}$. Полагая, что $y=x-1$ имеем: $-\lim\limits_{y\to0}\dfrac{\cos\frac{\pi(y+1)}{2}(1+\sqrt{y+1})}{y}=$
	
	$
	= \lim\limits_{y\to0} \dfrac{\sin \frac{\pi y}{2}(1+\sqrt{y+1})}{y} = \dfrac{\pi}{2}\lim\limits_{y\to0}\dfrac{\sin \frac{\pi y}{2}(1+\sqrt{y+1})}{\frac{\pi y}{2}}  = \pi
	$
	
	{\bf 239.} $\lim\limits_{x\to0} \dfrac{1-\sqrt{\cos x}}{x^2}= \lim\limits_{x\to0} \dfrac{1-\cos x}{x^2(1+\sqrt{\cos x})} = \lim\limits_{x\to0} \dfrac{1-\cos^2 x}{x^2(1+\sqrt{\cos x})(1+\cos x)} = $
	
	$
	=\lim\limits_{x\to0} \dfrac{\sin^2 x}{x^2(1+\sqrt{\cos x})(1+\cos x)} = \lim\limits_{x\to0} \dfrac{1}{(1+\sqrt{\cos x})(1+\cos x)}= \dfrac{1}{2\cdot2} = \dfrac{1}{4}
	$
	
	\medskip
	{\bf 240.} $\lim\limits_{x\to0}\dfrac{\sqrt{1+\sin x}-\sqrt{1-\sin{x}}}{x} = \lim\limits_{x\to0} \dfrac{1+\sin x-1+\sin x}{x(\sqrt{1+\sin x}+\sqrt{1-\sin x})} = \lim\limits_{x\to0} \dfrac{2\sin x}{x(\sqrt{1+\sin x}+\sqrt{1-\sin x})} = $
	
	$
	=\lim\limits_{x\to0} \dfrac{2}{\sqrt{1+\sin x}+\sqrt{1-\sin x}} = \dfrac{2}{\sqrt{1}+\sqrt{1}} = 1
	$
	
	
	
	
	
	
	
	
	
	
	
	
	
	
	
	
	
	
	
	
	
	
	
	
	
	
	
	
	
	
	
	
	
	
	
	
	
	
	
	
\end{document}